{\singlespacing
   \begin{flushright}
      William E. Wright \\
      November 2019 \\
      Mathematics \\
   \end{flushright}
}

\bigskip

\begin{center}
   An Improved Gauge Dual Descent Algorithm for Noisy Phase Retrieval \\
\end{center}

\section*{Abstract}

Phase retrieval is a common problem in signal processing, with applications in astronomy, x-ray imaging, electron microscopy, and coherent diffraction imaging (CDI).
Many models and algorithms exist for phase retrieval,
yet few are designed to handle noise without imposing additional assumptions like signal sparsity.  
One recent algorithm for noisy phase retrieval which requires no underlying assumptions is the Gauge Dual Descent (GDD) algorithm, which iteratively denoises the desired signal.

The GDD algorithm involves a sequence of eigenvalue problems which can be computationally expensive for large-scale signal recovery, requiring many matrix-vector products.
Additionally, signal denoising progress tends to stall in the GDD algorithm prior to satisfying optimality-based termination conditions.
%Additionally, the GDD algorithm typically fails to meet previously proposed termination conditions based on gauge dual optimality.
To address these challenges, this dissertation provides the following contributions.
First, we establish empirical termination conditions for the GDD algorithm which suggest signal denoising progress has stalled.
Next, we develop an adaptive strategy for handling the sequence of eigenvalue problems in the GDD algorithm.
These contributions lead to the Improved Gauge Dual Descent (IGDD) algorithm.
Numerical examples demonstrate that the IGDD algorithm requires $50-80\%$ fewer matrix-vector products than the GDD algorithm for a variety of problems with low-oversampling.
